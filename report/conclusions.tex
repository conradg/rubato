\chapter{Conclusions}
This chapter summarises the achievement of this project, as well as pointing out ideas for future work.

Adaptive learning in the context of music education is an unexplored field, as adaptive learning is still a growing field itself. In this project, I have devised a computer application that provides an adaptive learning experience for intonation training using a segmented autocorrelation algorithm. 
Initially, I hoped to include several different exercises that would test a wide variety of musical skills, however, I have only managed to get the intonation exercise up and running.
The exercise itself is, I believe, a successful one, and one unlike anything else existing. 
Adaptive learning has been successfully implemented in this exercise, and when a user spends time on it, they will find that it changes to match their ability.
The pitch detection algorithm is highly successful and reliable, and it has shown that it can cope with variable levels of background noise.



\section{Future}
Though the project is finished, there is much more that could be done to enhance and extend it. \begin{itemize}
	\item A greater variety of exercises. At present, the only exercise available to the user is interval training. There are many different exercises that could work in an adaptive learning context, for example, the rhythmical exercise mentioned in the background.
	\item A melody analyser. The pitch detection algorithm is nearly sufficiently sophisticated as to be able to detect pitch in samples containing melodies. This would be an interesting feature that could pave the way for automated music dictation.

\end{itemize}