\chapter{Introduction}

\par In this paper I propose an application to harness the power of the modern web browser to deliver engaging musicality tutoring based on adaptive learning principles. The application will comprise of various exercises designed to challenge and improve a user's musicality, and keep track of their progress as well as adapt to their strengths and weaknesses in different areas.
\vspace{1em}
\par
As usage of the world wide web has become ubiquitous among citizens of the developed world, web based teaching has grown rapidly to try and modernize learning methods to fit into the 21st century lifestyle. Increasing numbers of companies like Coursera\cite{coursera}, Codeacademy\cite{codeacademy}, Duolingo\cite{duolingo} are providing easily accessible education for anyone with an internet connection and a desire to learn. Codeacademy and Duolingo in particular are notable for their use of rich, highly interactive learning tools which - through requiring the user to have an input into the educational process - establish a feedback system with powerful results\cite{vesselinov2012duolingo}. These companies offer a variety of different programmes the user can study, and each of these is taught through a series of exercises, the results of which are used to track the users progress through the course, and provide statistics on how much they have learned. Duolingo, a website that teaches foreign languages, takes it a step further by introducing adaptive learning methods to recognise the user's proficiency in various areas, information it can then use to tailor-make lessons to fit the user's needs. Duolingo is arguably the best best-known example of an adaptive learning system put into practice. It will therefore often be used for comparison throughout this report, as it has also provided a lot of inspiration for my thinking about the direction my project will take.
\vspace{1em}
\par
The web is a great place to teach music theory, and train the musical ear, a pair of skills we shall refer to under the umbrella term of "musicality" from hereon in. The rich media options possible on modern day web browsers mean that the input and output of music to a web browser is not only easy to implement, but easy to make user friendly. However, while there are many "music theory tutors" and "ear trainers" available, there are no existing adaptive learning solutions. Such a solution would hopefully be invaluable to potential learners as the current best solutions still have no idea who you are and what you've achieved after you leave the page.
\par

In order to teach adaptively we must analyse information about a user's performance, and then change how we teach accordingly. 
There are various ways of measuring musicality, which will be discussed in more detail later, and we will rely on these metrics to adapt the learning experience to fit the user.



\section{Objectives}

I aim to build a program that can successfully teach musicality. To ensure I achieve this goal, the following criteria must be met:
	\begin{itemize}
		\item \textbf{Adaptive} - The product must analyse the user's progress and their strengths, and adopt the content of their learning experience accordingly. This will require a user account system, as well as intelligent handling of the exercise data they generate.
		\item \textbf{Intuitive} - The product must be simple to use, and the exercises given to the user must be simple to understand, even for a beginner.
		\item \textbf{Engaging} - The user must want to learn and continue learning. In a study of Duolingo's effectiveness, 93.8\% of participants intended to continue using the website after the study had finished \cite{griffiths2012profile}. I would like to aim for at least 75\%.
		\item \textbf{Pitch Recognition} - Another way I'm aiming push the boundaries of musicality teaching is through introducing exercises that involve pitch data to be submitted by the user, and then grade them based on how accurate they are.
	\end{itemize}