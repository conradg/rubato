\chapter{Evaluation}

In order to evaluate the success of my program, I have used both qualitative and quantitative methods which I will outline in the chapter.

\section{Adaptivity}
How well I have achieved adaptivity in my program has been a difficult thing to evaluate, it is hard to quantify how well someone has been taught, and to achieve a sample size of students anywhere near what you would need to meaningful perform quantitative analysis of their performance has been difficult. I have, however managed to ask a few musical, and non-musical volunteers to use the program and give me their feedback.

Among more musically experienced volunteers, the feedback was good. Users reported they felt challenged by the exercises, and they noticed when the program increased or decreased it's difficulty.

For the less musically experienced volunteers, the feedback was less positive. Some of the starting levels of the exercises were reported to be too difficult, and they felt like they didn't know what was meant to be expected of them.

\section{Pitch Detection}

Robiner, Cheng, Rosenberg, and McGonegal's work in analysing pitch detectors for speech analysis give several different criteria for evaluation of a pitch detector\cite{comparativePitchDetection}, and we are going to use 3 of them to evaluate ours. They are:
	\begin{enumerate}
		\item Accuracy in estimating pitch period
		\item Robustness of measurements. Does the pitch detector hold up in different environments other than the one it was developed in? Different voice types, and microphones could have an impact on this.
		\item Speed of operation


	\end{enumerate}
	
\subsection{Period accuracy}
To test the period accuracy of the detector it will be necessary to compare it's output to a source with a known period. I have not been able to find any existing banks of data that I can use for this, so I have come up with my own\cite{testAudio}.

The test set consists of many different samples of 1 or 2 seconds of a note being sung or played on the piano at a known pitch. I have included samples from multiple different people in my test set to make sure my pitch detection works for as many different voice types as possible. I have also recorded people using a very basic microphone inbuilt on my laptop, so I can ensure that the pitch detection will work even with fairly low quality audio. 

The vocal test set includes 70 samples of both male and female voice singing a range of different notes The piano test set includes 36 samples of notes being played over a 3 octave range, from a G2 to G5. A link is available in the bibliography to a zip file containing all of these samples in WAV format\cite{testAudio}.

For both the piano samples and the vocal samples, the pitch detector was 100\% accurate in detecting the pitch to the nearest semitone. It is difficult to say with any greater degree of certainty how accurate the voice sample detection accuracy was, as we can't be entirely sure whether it is the voice that is very subtly singing the wrong note, or the detector detecting the wrong period. For the piano however, we can examine the results in  greater detail, as we can be more certain of the correctness of the pitch. The results are seen in table below.
Firstly, we can calculate the sample standard deviation of the cents given by the following formula:
\[s=\sqrt{\frac{1}{N-1}\sum_{i=1}^{N} (x_i - \bar{x}^2)}\]

Where \(N\) is the the sample size, and \(\bar{x}\) is the mean of the samples. This gives us a value of 10.180. To put this into perspective, the human ear is estimated to be able to distinguish differences of pitch of about 5-6 cents\cite{loeffler2006instrument}, and a semitone is 100 cents, so this is an acceptable range for the standard deviation.

Interestingly, the mean of this data is 3.081 cents, implying that on average, our pitch detection is a bit sharp. It is unknown whether this is a function of some acoustical phenomenon on the piano, or whether this is a problem with all detection. The pitch detector has been used to detect computer generated sine waves as well, which it achieved with very high precision, and no sharp/flat bias, so we can rule out some systematic mathematical bias caused by incautious rounding of a variable, or something like that.

\subsubsection{Piano results}
\begin{center}
	\begin{tabular}{| l | l | l |}
	\hline
	Expected pitch & Detected pitch & Cent error \\ \\ \hline
	G2  & G2  & 10  \\ \hline
	Ab2 & Ab2 & 4  \\ \hline
	A2  & A2  & 5  \\ \hline
	Bb2 & Bb2 & 6  \\ \hline
	B2  & B2  & -3  \\ \hline
	C3  & C3  & 1  \\ \hline
	C\#3 & C\#3 & 5  \\ \hline
	D3  & D3  & 7  \\ \hline
	Eb3 & Eb3 & 4  \\ \hline
	E3  & E3  & 4  \\ \hline
	F3  & F3  & 2  \\ \hline
	F\#3 & F\#3 & -12  \\ \hline
	G3  & G3  & -5  \\ \hline
	Ab3 & Ab3 & 3  \\ \hline
	A3  & A3  & 3  \\ \hline
	Bb3 & Bb3 & 9  \\ \hline
	B3  & B3  & 4  \\ \hline
	C4  & C4  & 6  \\ \hline
	C\#4 & C\#4 & 7  \\ \hline
	D4  & D4  & 8  \\ \hline
	Eb4 & Eb4 & 4  \\ \hline
	E4  & E4  & 1  \\ \hline
	F4  & F4  & 5  \\ \hline
	F\#4 & F\#4 & 1  \\ \hline
	G4  & G4  & 3  \\ \hline
	Ab4 & Ab4 & 3  \\ \hline
	A4  & A4  & -7  \\ \hline
	Bb4 & Bb4 & 0  \\ \hline
	B4  & B4  & 4  \\ \hline
	C5  & C5  & 2  \\ \hline
	C\#5 & C\#5 & 6  \\ \hline
	D5  & D5  & 10  \\ \hline
	Eb5 & Eb5 & 32  \\ \hline
	E5  & E5  & 23  \\ \hline
	F5  & F5  & 5  \\ \hline
	F\#5 & F\#5 & -8  \\ \hline
	G5  & G5  & -38  \\ \hline
	\end{tabular}
\end{center}

\subsection{Robustness}
The pitch detector was tested in various different environments, such as on different laptops, and in environments with high background noise. As the autocorrelation is quite noise tolerant, we didn't find a significant change in performance when tested with high background noise.

\subsection{Speed}
The speed of the pitch detector was evaluated over the test set, and determined to be 0.207 seconds per sample. This is fast enough to not cause the user to wait around, so this is a satisfactory measurement for our purposes.