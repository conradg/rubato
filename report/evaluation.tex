\chapter{Evaluation}

In order to evaluate the success of my program, I will need to analyse each of my objectives to see how far I've come in acheiving them.

\section{Adaptivity}
How well I have achieved adaptivity in my teaching will be a difficult thing to test quantitatively, as there is no existing way of measuring how adaptive a learning system is. One way to gain a greater understanding of how effective the adaptivity is would be to use the program and artificially fail parts of exercises to try and prompt the system to adapt to my behaviour, and see how it responds.

\section{Intuitiveness}
The intuitiveness of my program is something that can only be judged by other people using it. I intend to involve others in the design process from the start, by using techniques like hallway testing to quickly get feedback on how users approach my app, and what they would change about it. At the end of the project I will also hopefully get as many different people as I can to use it, and fill in a survey at the end of their usage period detailing their experience with the app.
\section{User Engagement}
This will be handled by users answering questions about whether they would continue using it, and how much they enjoyed the process. These questions will be found on the survey described above.

\section{Teaching Profficiency}
To measure the general success of my product as a teaching tool, I shouldn't actually have to do too much, as (if succesfully implemented), my app should be able to track the users progress, and store information about how much they've improved.

\section{Pitch Detection}
Pitch detectors are prone to error, so it is crucial to evaluate their performance thoroughly. They can be thrown off for many reasons, including varied vowel sound, a different person's voice, and background noise, so it will be essential to test the detector in different environments that the app may face. Determining automatically the success of pitch detection algorithms requires you to be able to measure pitch correctly in the first place, a somewhat circular reference. Therefore the only way to do it will be to manually try out different environments, and measure how successfully the app classifies pitch in each situation. I will also try and find multiple people with different voice types to help test the pitch detection to further try and ensure it is as \bsq{context-free} as possible. This will be a fairly arduous task that will need to be ongoing throughout the development process.