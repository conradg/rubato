\chapter{Evaluation}

In order to evaluate the success of my program, I have used both qualititative and quantitative methods which I will outline in the chapter.

\section{Adaptivity}
How well I have achieved adaptivity in my teaching will be a difficult thing to test quantitatively, as there is no existing way of measuring how adaptive a learning system is. One way to gain a greater understanding of how effective the adaptivity is would be to use the program and artificially fail parts of exercises to try and prompt the system to adapt to my behaviour, and see how it responds.

\section{Intuitiveness}
The intuitiveness of my program is something that can only be judged by other people using it. I intend to involve others in the design process from the start, by using techniques like hallway testing to quickly get feedback on how users approach my app, and what they would change about it. At the end of the project I will also hopefully get as many different people as I can to use it, and fill in a survey at the end of their usage period detailing their experience with the app.
\section{User Engagement}
This will be handled by users answering questions about whether they would continue using it, and how much they enjoyed the process. These questions will be found on the survey described above.

\section{Teaching Proficiency}
To measure the general success of my product as a teaching tool, I shouldn't actually have to do too much, as (if succesfully implemented), my app should be able to track the users progress, and store information about how much they've improved.

\section{Pitch Detection}
I have developed a test set for my pitch detection algorithm. This consists of multiple different samples of 1 or 2 seconds of a note being sung at a specific known pitch. I have included samples from many different people in my test set to make sure my pitch detection works for as many different voice types as possible. I have also recorded people using a very basic microphone used on my laptop, so I can ensure that the pitch detection will work even with fairly low quality audio. 

The test set includes 70 samples of both male and female voice, a link is available in the bibliography to a zip file containing these samples in WAV format.


following sections includes pictures of my pitch detection output, as well as percentage correct results for my algorithm.